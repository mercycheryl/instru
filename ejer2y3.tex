\documentclass[a4paper,12pt]{article}
\usepackage[utf8]{inputenc}
\usepackage{amsmath}
\usepackage{geometry}
\geometry{margin=1in}
\title{Ejercicios de Instrumentación Industrial}
\author{}
\date{}

\begin{document}

\maketitle

\section*{Ejercicio 2: Sensor capacitivo y caudalímetro ultrasónico}

\subsection*{Datos:}
\begin{itemize}
    \item Sensor capacitivo: cilindros de aluminio de 8 cm (Dext) y 6 cm (Dint)
    \item Longitud del sensor: 8 m
    \item Capacidad medida: 3055.63 pF
    \item Aceite SAE 10:
    \begin{itemize}
        \item Densidad relativa: 0.877
        \item Constante dieléctrica relativa: 2.3
    \end{itemize}
    \item Tubería:
    \begin{itemize}
        \item Diámetro: 150 mm
        \item Pérdidas totales: 3 m
    \end{itemize}
    \item Velocidad del sonido en aceite: 1770 m/s
    \item $\varepsilon_0 = 8.854$ pF/m
\end{itemize}

\subsection*{Solución:}

\begin{enumerate}
    \item Cálculo del nivel de aceite ($h$):
    \[
    C = \frac{2 \pi \varepsilon_0 \varepsilon_r h}{\ln(D_{ext}/D_{int})}
    \]
    Despejando $h$:
    \[
    h = \frac{C \ln(D_{ext}/D_{int})}{2 \pi \varepsilon_0 \varepsilon_r}
    \]
    Sustituyendo valores:
    \[
    h = \frac{3055.63 \times \ln(8/6)}{2 \pi \times 8.854 \times 2.3} \approx 4.5 \, \text{m}
    \]

    \item Cálculo del caudal ($Q$):
    \[
    \frac{v^2}{2g} + h = \text{pérdidas} \implies v = \sqrt{2g \times 3} \approx 7.67 \, \text{m/s}
    \]
    Caudal:
    \[
    Q = v \times A = 7.67 \times \pi \times (0.075)^2 \approx 0.135 \, \text{m}^3/\text{s}
    \]

    \item Diferencia de tiempo en caudalímetro ultrasónico:
    
    Para $\theta = 45^\circ$ y $L = 0.15$ m:
    \[
    \Delta t = \frac{2L \cos \theta}{c^2} v = \frac{2 \times 0.15 \times \cos 45^\circ}{1770^2} \times 7.67 \approx 5.2 \, \mu \text{s}
    \]
\end{enumerate}

\section*{Ejercicio 3: Barra magnetostrictiva e inductancia}

\subsection*{Datos:}
\begin{itemize}
    \item Fuerzas aplicadas: 20 kN (compresión), 15 kN (tracción)
    \item Dimensiones:
    \begin{itemize}
        \item Aluminio: 700 mm, $A = 700$ mm²
        \item Bronce: 500 mm, $A = 1000$ mm²
        \item Magnetostrictivo: 600 mm, $A = 800$ mm²
    \end{itemize}
    \item Propiedades magnéticas:
    \begin{itemize}
        \item $\mu_r$ varía linealmente de 1500 a 9800 para $\sigma = 5$ a 40 MPa
        \item Bobina: 300 espiras
        \item $\mu_0 = 4 \pi \times 10^{-7}$ H/m
        \item $\mu_r$ acero = 1950
    \end{itemize}
\end{itemize}

\subsection*{Solución:*
