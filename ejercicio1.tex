\documentclass[a4paper,12pt]{article}
\usepackage[utf8]{inputenc}
\usepackage{amsmath}
\usepackage{geometry}
\geometry{margin=1in}
\title{Solución Detallada del Problema 1}
\author{}
\date{}

\begin{document}

\maketitle

\section*{1a. Lectura del PT100 en la superficie externa de la pared}

\subsection*{Datos:}
\begin{itemize}
    \item NTC: 47 k$\Omega$, $\beta = 2420$
    \item Voltaje de salida sobre el NTC: 3.97 V
    \item Flujo de calor: 2250 W/m²
    \item Coeficiente de convección ($h$): 75 W/m²·K
    \item Emisividad ($\varepsilon$): 0.8
    \item Constante de Boltzmann ($\sigma$): $5.67 \times 10^{-8}$ W/m²·K$^4$
    \item PT100: $\alpha = 0.00385 \, \Omega/\Omega^\circ$C
\end{itemize}

\subsection*{Procedimiento:}

\begin{enumerate}
    \item Calcular la resistencia del NTC:
    \[
    R_{NTC} = 47 \, k\Omega \times \left(\frac{5}{3.97}\right) - 47 \, k\Omega \approx 12.2 \, k\Omega
    \]

    \item Determinar la temperatura del NTC:
    \[
    \frac{1}{T} = \frac{1}{T_0} + \frac{1}{\beta} \ln\left(\frac{R_{NTC}}{R_0}\right)
    \]
    Con $T_0 = 298$ K (25°C):
    \[
    \frac{1}{T} = \frac{1}{298} + \frac{1}{2420} \ln\left(\frac{12.2}{47}\right) \implies T \approx 323 \, K \, (50^\circ C)
    \]

    \item Balance de energía en la superficie externa:
    \[
    q_{conducción} = q_{convección} + q_{radiación}
    \]
    \[
    2250 = 75(T_s - 323) + 0.8 \times 5.67 \times 10^{-8} (T_s^4 - 323^4)
    \]
    Resolviendo iterativamente, $T_s \approx 353$ K (80°C).

    \item Resistencia del PT100:
    \[
    R_{PT100} = 100 \, \Omega \times (1 + 0.00385 \times 80) \approx 130.8 \, \Omega
    \]

    \item Voltaje de salida del puente:
    \[
    V_{out} = 10 \times \left(\frac{130.8}{130.8 + 100} - 0.5\right) \approx 0.67 \, V
    \]
\end{enumerate}

\section*{1b. Lectura de la termocupla tipo J en la superficie interna}

\subsection*{Datos:}
\begin{itemize}
    \item Conductividad térmica del ladrillo ($k$): 1.2 W/m·K
    \item Ancho de la pared ($L$): 2.54 m
    \item Temperatura de referencia: 28°C
\end{itemize}

\subsection*{Procedimiento:}

\begin{enumerate}
    \item Calcular la temperatura interna ($T_i$):
    \[
    q = k \frac{T_i - T_s}{L}
    \]
    \[
    2250 = 1.2 \times \frac{T_i - 353}{2.54} \implies T_i \approx 823 \, K \, (550^\circ C)
    \]

    \item Fuerza electromotriz (EMF) de la termocupla tipo J:
    
    Según tablas de termocuplas para $T_i = 550^\circ$C y $T_{ref} = 28^\circ$C:
    \[
    EMF \approx 28.5 \, mV
    \]
\end{enumerate}

\section*{Respuestas Finales:}
\begin{itemize}
    \item \textbf{a.} La lectura del PT100 es \textbf{130.8} $\Omega$, con un voltaje de salida de \textbf{0.67} V.
    \item \textbf{b.} La lectura de la termocupla tipo J es \textbf{28.5} mV.
\end{itemize}

\end{document}
